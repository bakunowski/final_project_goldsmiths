\chapter{Introduction}

\section{Aims and Objectives}

The initial aim of the project was to implement a machine learning solution to
the task of granular synthesizer programming based on sound matching.
Consequently building a tool that would assist musicians in creating interesting
sounds, provided an audio input to the system.

In the process of building this tool, the project extended into more
technical direction, with a focus of creating a usable tool, despite
some shortcomings in the areas of machine learning. More specifically,
a large amount of time was spent configuring the three main pieces of
the project: the synthesizer, audio analysis and machine learning to
work together as smoothly as possible, rather than making each of them
exceptional on it's own.

The methods here differ from other approaches in literature (cite some
papers that have done this, eg. Matthew, Leon) primarily in that here,
one standalone tool has been created, that is usable on it's own,
directly by the user. A usable and focus directly on the interaction between
user and the machine. Direct access to predictions and straightforward
usability, handled by two buttons.

The main objectives are:

1. To build a tool that is helpful to artists in creating new sounds 

2. To challenge the interaction between an artist and a preset as a starting
point to synthesis 

3. To achieve a response that is not only stimulating to the user but also differs
from simply randomizing the parameter values 

In measuring the project's success, the subjective sonic coherence and
similarity of predicted outputs may be considered the best indicator (the
human discriminator?). Along with that, a more quantitative analysis, such as comparing
the audio descriptors on target and predicted sounds should prove itself useful,
as it will allow to do a quite generalisable, and objective comparison of target
vs predicted sounds.  

In (future chapters) I provide some critique on established techniques of
assessment of these types of problems and ultimately conclude that some stuff in
the best way of doing it.

\subsection{Deliverables}

To achieve that, I propose a standalone application, that combines all three
main aspects, and lets the user set parameters of a synthesizer based purely on
the microphone input to the system.  Synthesis, implemented with the `JUCE'
framework.  Audio analysis tools, implemented with the help of the `Essentia'
library, and an implementation of a Multilayered Perceptron feedforward neural
network, done in `Keras'. Together with the help of the `Frugally Deep' library,
responsible for deploying the Keras model into C++ code, the three modules were
able to work together inside of the synthesis program.
% \cite{example-citation} Some more things. 
% Inline citation: \bibentry{example-citation}


%This section should be about 500 words.
%
%You can't write a good introduction until you know what the body of
%the paper says. Consider writing the introductory section (s) after you have completed the rest of the paper, rather than before.
%
%Be sure to include a hook at the beginning of the introduction. This is a statement of something sufficiently interesting to motivate your reader to read the rest of the paper, it is an important/interesting scientific problem that your paper either solves or addresses. You should draw the reader in and make them want to read the rest of the paper.
%
%The next paragraphs in the introduction should cite previous research in this area. It should cite those who had the idea or ideas first, and should also cite those who have done the most recent and relevant work. You should then go on to explain why more work was necessary (your work, of course.)
% 
%What else belongs in the introductory section (s) of your paper? 
%
%1.    A statement of the goal of the paper: why the study was undertaken, or why the paper was written. Do not repeat the abstract. 
%
%2.   Sufficient background information to allow the reader to understand the context and significance of the question you are trying to address. 
%
%3. Proper acknowledgement of the previous work on which you are building.
%Sufficient references such that a reader could, by going to the library, achieve
%a sophisticated understanding of the context and significance of the question.
%
%4.    The introduction should be focused on the thesis question(s).  All cited work should be directly relevant to the goals of the thesis.  This is not a place to summarize everything you have ever read on a subject.
%
%5.    Explain the scope of your work, what will and will not be included. 
%
%6.    A verbal `road map' or verbal `table of contents' guiding the reader to what lies ahead. 
%
%7.    Is it obvious where introductory material (`old stuff') ends and your contribution (`new stuff') begins? 
%
%Remember that this is not a review paper. We are looking for original work and interpretation/analysis by you. Break up the introduction section into logical segments by using subheads. 