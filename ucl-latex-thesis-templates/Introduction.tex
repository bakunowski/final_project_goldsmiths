\chapter{Introduction}
\label{chapterlabel1}

\section{Aims and Objectives}
%i should consider writing a 'hook' sentence/paragraph here to draw readers
%attention and make them interested in the rest of the paper
The initial aim of the project was to implement a machine learning solution to
the task of granular synthesizer programming based on sound matching.
Consequently building a tool that would assist musicians in creating interesting
sounds, provided an audio input to the system.

Over the course of the past couple months the project extended into more
research-oriented directions, with a focus of finding the best possible audio
descriptors for specific characteristics of sounds. In order to determine the
most significant in judging similarity on the cognitive level.

The methods here concerned with describing audio differ from other approaches in
literature (e.g Matthew) primarily in that sound is treated in a modular manner,
where a description is made based on one characteristic for X number instances,
and separate processes are run for each. Instead of trying to describe audio as
a whole, the aim is to describe it as a combination of things (density, pitch,
rhythm, amplitude)

Concretely my objectives are:

1. To build a tool that is helpful to artists in creating new sounds 

2. To challenge the interaction between an artist and a preset as a starting
point to synthesis 

3. To achieve a response that is not only stimulating to the user but also differs
from simply randomizing the parameter values 

In measuring the project's success, the subjective sonic coherence and
similarity of algorithm's outputs may be considered the best indicator (the
human discriminator?), along with more quantitative analysis, such as comparing
the audio descriptors on input and output sounds for example.

In (future chapters) I provide some critique on established techniques of
assessment of these types of problems and ultimately conclude that some stuff in
the best way of doing it.

\subsection{Deliverables}
to these ends hehhe this is what i made:

Some stuff about things.\cite{example-citation} Some more things. 
Inline citation: \bibentry{example-citation}

\iffalse
This section should be about 500 words.

You can't write a good introduction until you know what the body of the paper says. Consider writing the introductory section(s) after you have completed the rest of the paper, rather than before.


Be sure to include a hook at the beginning of the introduction. This is a statement of something sufficiently interesting to motivate your reader to read the rest of the paper, it is an important/interesting scientific problem that your paper either solves or addresses. You should draw the reader in and make them want to read the rest of the paper.

The next paragraphs in the introduction should cite previous research in this area. It should cite those who had the idea or ideas first, and should also cite those who have done the most recent and relevant work. You should then go on to explain why more work was necessary (your work, of course.)
 
What else belongs in the introductory section(s) of your paper? 

1.    A statement of the goal of the paper: why the study was undertaken, or why the paper was written. Do not repeat the abstract. 

2.   Sufficient background information to allow the reader to understand the context and significance of the question you are trying to address. 

3. Proper acknowledgement of the previous work on which you are building.
Sufficient references such that a reader could, by going to the library, achieve
a sophisticated understanding of the context and significance of the question.

4.    The introduction should be focused on the thesis question(s).  All cited work should be directly relevant to the goals of the thesis.  This is not a place to summarize everything you have ever read on a subject.

5.    Explain the scope of your work, what will and will not be included. 

6.    A verbal "road map" or verbal "table of contents" guiding the reader to what lies ahead. 

7.    Is it obvious where introductory material ("old stuff") ends and your contribution ("new stuff") begins? 

Remember that this is not a review paper. We are looking for original work and interpretation/analysis by you. Break up the introduction section into logical segments by using subheads. 
\fi