\chapter{Literature Review}
\label{chapterlabel2}

\section{Problem background}
%VERY CONCISE 
Programming synthesizers is a fun, but challenging task. The range of sounds possible to
achieve on such a device, and consequently the amount of adjustable parameters can be overwhelming.
From choosing an audio file to sample, to an amplitude envelope for each grain, the ability to make
decisions about programming a granulator has to come from either a place of certainty about what
each parameter is responsible for, or a place of experimental thought, and a somewhat random
parameter value assignments

Users fairly new in the realm of synthesizer programming may encounter issues creating sounds they
desire. Even successful musicians might be doing things following a not clearly defined “intuition”.
There are of course people who are experts in this field, but artists often look for inspiration
when it comes to timbre of their sounds%\cite{noauthor_oneohtrix_2016} \cite{herbert_manifesto_2011}.
It would seem that, the use of presets as a starting point is not unusual.

%My future thesis will aim to offer an alternative way to approach the creation of new sounds.

%based on this write sections i think
Research has been done previously as an investigation into automation of parameters in synthesizers
based on sound matching. Taking a snippet of sound, the algorithm would try to
find parameter settings to match a produced sound as closely as possible to the
source%\cite{yee-king_automatic_2018}.

This research mostly focuses on FM synthesis
% \cite{horner_machine_1993}
, although
experiments on different synthesis techniques has been
done
% \cite{dahlstedt_creating_nodate}
, including any VST
plug-in
% \cite{yee-king_synthbot:_nodate}.

However, using this approach on corpus-based synthesis remains an untapped area
of research, worth investigating
%\cite{mcdonald_neural_2017}.

Also, it seems that most of this research focuses on reproducing the original
input
% \cite{tatar_automatic_2016}
. Perhaps more interesting and novel sounds
could arise as an effect of bad performance, but is does not seem to be the
desired outcome in most cases.

\subsection{Audio descriptors}
\subsection{Granulation}
\subsection{Machine Learning}

\iffalse
It seems like a description of this section is also in the introduction
description. Which means that it could probably be fitted in there, instead of
making a whole another section for it in the paper.
\fi