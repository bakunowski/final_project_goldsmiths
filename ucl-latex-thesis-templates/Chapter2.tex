\chapter{Literature Review}
\label{chapterlabel2}

\section{Problem background}
%VERY CONCISE 
- what is the problem

- some study that describes it directly, and is a general thing done on the main
problem here - programming synths based on sound matching

\subsection{Audio descriptors}
MFCC - description - most popular in speech synthesis

what other can i mention that are useful, and widely used, but not as popular?
mention the ones that i think will be useful in this project - paired with a granulator.

\subsection{Granulation}
difficult problem as the sound heavily depends on the audio sampled

to accommodate for that - some universal audio descriptor have to be used that
describe rhythm, density, pitch etc - so that the input sound can be 'molded'
into what will resemble the original input sound to the system. 

concatenative synthesis - how it could be beneficial to make that - at least
mention it and some studies about it. mention CataRT maybe...

\subsection{Machine Learning}
a neural network could me used to build a granular synthesis parameter space
from a sound, but teaching it to do that, and predicting it's behaviour is a
difficult task as well as a very experimental approach, that is quite difficult
to controll and predict.

instead, focusing on replicating certain aspects of the input sound seems like a
more straightforward and predictable approach to this problem.

slicing the machine learning algorithm to predict different aspects of the sound
separately will result in a more predictable outcome as well as one that's
easier to control.




\iffalse
Programming synthesizers is a fun, but challenging task. The range of sounds possible to
achieve on such a device, and consequently the amount of adjustable parameters can be overwhelming.
From choosing an audio file to sample, to an amplitude envelope for each grain, the ability to make
decisions about programming a granulator has to come from either a place of certainty about what
each parameter is responsible for, or a place of experimental thought, and a somewhat random
parameter value assignments.

Users fairly new in the realm of synthesizer programming may encounter issues creating sounds they
desire. Even successful musicians might be doing things following a not clearly defined “intuition”.
There are of course people who are experts in this field, but artists often look for inspiration
when it comes to timbre of their sounds.
%\cite{noauthor_oneohtrix_2016} \cite{herbert_manifesto_2011}
It would seem that, the use of presets as a starting point is not unusual.

%My future thesis will aim to offer an alternative way to approach the creation of new sounds.

%based on this write sections i think
Research has been done previously as an investigation into automation of parameters in synthesizers
based on sound matching. Taking a snippet of sound, the algorithm would try to
find parameter settings to match a produced sound as closely as possible to the
source%\cite{yee-king_automatic_2018}.

This research mostly focuses on FM synthesis
% \cite{horner_machine_1993}
, although
experiments on different synthesis techniques has been
done
% \cite{dahlstedt_creating_nodate}
, including any VST
plug-in
% \cite{yee-king_synthbot:_nodate}.

However, using this approach on corpus-based synthesis remains an untapped area
of research, worth investigating
%\cite{mcdonald_neural_2017}.

Also, it seems that most of this research focuses on reproducing the original
input
% \cite{tatar_automatic_2016}
. Perhaps more interesting and novel sounds
could arise as an effect of bad performance, but is does not seem to be the
desired outcome in most cases.


It seems like a description of this section is also in the introduction
description. Which means that it could probably be fitted in there, instead of
making a whole another section for it in the paper.

The next paragraphs in the introduction should cite previous research in this
area. It should cite those who had the idea or ideas first, and should also cite
those who have done the most recent and relevant work. You should then go on to
explain why more work was necessary (your work, of course.)

Sufficient background information to allow the reader to understand the context
and significance of the question you are trying to address.

Proper acknowledgement of the previous work on which you are building.
Sufficient references such that a reader could, by going to the library, achieve
a sophisticated understanding of the context and significance of the question.

The introduction should be focused on the thesis question(s). All cited work
should be directly relevant to the goals of the thesis. This is not a place to
summarize everything you have ever read on a subject.

\fi