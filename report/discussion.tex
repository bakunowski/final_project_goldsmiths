\chapter{Discussion}
\label{chapterlabel5}

\section{Summary of Findings}
To summarise what I concluded (via either qualitative or quantitative
analysis) in the last chapter:
\begin{itemize}
\item The implementation of the MFCCs, neural network and the synthesizer are sound
\item The MFCCs in sounds predicted on output of the synthesizer
    itself are very close to the target MFCCs.
\item The Euclidean distance of MFCCs predicted on from the microphone
  input is much bigger than in predictions made from synthesizer output 
\item In both cases however, the distance in much smaller than
  compared to MFCCs achieved through pure randomisation of parameters
\item Most of the testers agree that this tool has potential to be
  useful in the process of creation of new sounds  
\item 100\% of the testers think that this process would be a viable
  alternative to presets
\item On a scale from 0 to 5, the mean answer for similarity of sounds
  predicted is 3.8
\item Most tester agree that the program was intuitive to use
\end{itemize}

\section{Evaluation}
\subsection{Project objective 1: a helpful tool}

The overall opinion of all the people that have tested the program
seem to be that it would indeed be helpful for them in the process of
creating sound. There has been some suggestions about using it with in
a live environment, with a vocalist. Which leads to an interesting
extension of the project as well as a possibility for more research.

It would be possible to pre-train a model on certain samples, before a
live show, and then improvise with the use of that array of sounds,
live.

Most respondents agreed that they would be interested in implementing
such a program into their music production workflow. It would
certainly need more polishing, and work, yet already at it's current
state it can serve as an extension to music production arsenal.

\subsection{Project objective 2: alternative to presets}

In the questionnaire, 100\% of participants agreed that this workflow
is a viable alternative to present. One participant suggested, that
this interaction brings the user into the position of control, and
creates space for more control over the sound, than mindless scrolling
through presets. To quote:

``This project puts a little bit more responsibility with the user by
making them take a more active part in the creation of sounds rather
than scrolling through presets. Whilst presets can be a good way of
producing sounds I think that this makes users think more critically
about what sounds they are wanting to make and then use the tools
provided to replicate them.``

\subsection{Project objective 3: more than randomisation}

By measuring the distance between the MFCC vectors, as well as
repeated listening evaluation on my own, and with different users, it
is clear that sounds predicted by this system carry clear coorelation
with the target sounds. Less so with samples from the microphone
input, than the sounds created by the granulator, however still
satisfying to the end user. The result of this could be clearly
improved by spending more time tweaking the neural networks used, as
well as experimenting with separation of parameters and audio
features.

To summarize, the quantitative evaluation as well as user testing
shows that result obtained with this software are more than just a
randomisation of parameters.

\section{Future Work}

\subsection{Implementing more sophisticated neural networks}

\subsection{Creating a database of samples that can be used}

%%% Local Variables:
%%% mode: latex
%%% TeX-master: "dissertation"
%%% End:
