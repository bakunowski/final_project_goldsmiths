\chapter{Introduction}
\label{chapterlabel1}

\section{Aims and Objectives}
%i should consider writing a 'hook' sentence/paragraph here to draw readers
%attention and make them interested in the rest of the paper
The initial aim of the project was to implement a machine learning solution to
the task of granular synthesizer programming based on sound matching.
Consequently building a tool that would assist musicians in creating interesting
sounds, provided an audio input to the system.

Over the course of the past couple months, the project extended into a more
technical-oriented direction, implementing a granular synthesis algorithm in the
'JUCE' C++ framework. As well as into a research-oriented space, with a focus of
finding the best possible audio descriptors for specific characteristics of
sounds. In order to determine the most significant in judging similarity on the
cognitive level.

The methods here concerned with describing audio differ from other approaches in
literature
(SOURCES)%Matthew
primarily in that sound is treated in a modular manner,
where a description is made based on one characteristic for X number instances,
and separate processes are run for each. Instead of trying to describe audio as
a whole, the aim is to describe it as a combination of things (density, pitch,
rhythm, amplitude).

Concretely my objectives are:

1. To build a tool that is helpful to artists in creating new sounds 

2. To challenge the interaction between an artist and a preset as a starting
point to synthesis 

3. To achieve a response that is not only stimulating to the user but also differs
from simply randomizing the parameter values 

In measuring the project's success, the subjective sonic coherence and
similarity of algorithm's outputs may be considered the best indicator (the
human discriminator?), along with more quantitative analysis, such as comparing
the audio descriptors on input and output sounds for example.

In (future chapters) I provide some critique on established techniques of
assessment of these types of problems and ultimately conclude that some stuff in
the best way of doing it.

\subsection{Deliverables}

To these ends I have implemented 3 separate pieces of software:  

\begin{itemize}
\item Granular synthesizer implemented in the ``JUCE'' framework in C++
\item A module responsible for analysing audio input, and sending that data
\item A mixture of machine learing algorithms implemented in Python, that
        actually change synthesizer's parameters based on the input sound
\end{itemize}