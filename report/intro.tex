\chapter{Introduction}
\label{intro}
% %i should consider writing a 'hook' sentence/paragraph here to draw readers
% %attention and make them interested in the rest of the paper
\section{Aims and Objectives}

The initial aim of the project was to implement a machine learning
solution to the task of granular synthesizer programming based on
sound matching. Consequently building a tool that would assist
musicians in creating desired sounds more easily, provided an audio input to the system.

Throughout the entire development process, the focus remained mainly
on creating a well functioning, usable program, despite some
shortcomings in the areas of synthesis, audio analysis and machine
learning respectively. Integrating these three modules well took
priority over designing more sophisticated solutions to each
individual problem. This allowed for creation of a tool
functioning in near real-time, with the possibility of extending the
modules that contribute to the whole.

The methods here differ from other approaches in literature (cite some
papers that have done this, eg. Matthew, Leon) primarily in that here,
one standalone tool has been created, usable on it's
own. Users have direct ability to program the synthesizer using input
from the microphone, and the entire process is handled by two
buttons. No prior experience with programming is required, and the
only assumption made about the user is knowledge of basic synthesizer
programming.

The main objectives therefore are:

1. To build a tool that is helpful to artists in the process of
creating sounds 

2. To challenge the interaction between an artist and a preset as a
starting point to synthesis 

3. To achieve a response that is not only stimulating to the user but
also differs from simply randomizing the parameter values

In measuring the project's success, the subjective sonic coherence and
similarity of predicted outputs may be considered the best indicator (the
human discriminator?). Undoubtedly, if the predictions achieved are
satisfying to the user, the main goal of this project was
achieved. Additionally, users' opinions about whether the tool would
be usefull in their work process will be a good indicator of the
project's success.

Along with that, a quantitative analysis, such as comparing the audio
descriptors on target and predicted sounds should prove itself useful.
It will allow to do a quite generalisable, and objective comparison of
target vs. predicted sounds. This process will enable the evaluation
of neural networks used for predicting the parameter values, and help
in assuming whether users will find the program useful.  

EDIT!!!!!!:

In (future chapters) I provide some critique on established techniques of
assessment of these types of problems and ultimately conclude that some stuff in
the best way of doing it.

\subsection{Deliverables}

In the inerest of achieving the above decalred goals, I propose a
standalone program, that allows users to set parameters of a granular
synthesizer based on a one second buffer of audio, generated form the
microphone input.

The program could be arbitrarily split into three separate modules,
in order to clarify it's basic structure. Namely, synthesis, audio
analysis and machine learning.

The synthesis is implemented in the ``Juce'' framework. Audio analysis
tools from the ``Essentia'' library are used to help create training
datasets, as well as help make predictions. And lastly, an
implementation of a Multilayered Perceptron feedforward neural network
with the ``Keras'' API is presented. The ``Frugally Deep'' library is
responsible for deploying the ``Keras'' model into C++ code, allowing
for near-real time performance, and removing the need for
communication between C++ and Python. 

Resulting program is a standalone ``JUCE'' application, capable of
performing granluar synthesis, extracting audio features, and
predicting parameter values of the synthesiser based on microphone
input.

% \cite{example-citation} Some more things. 
% Inline citation: \bibentry{example-citation}

%This section should be about 500 words.
%
%You can't write a good introduction until you know what the body of
%the paper says. Consider writing the introductory section (s) after you have completed the rest of the paper, rather than before.
%
%Be sure to include a hook at the beginning of the introduction. This is a statement of something sufficiently interesting to motivate your reader to read the rest of the paper, it is an important/interesting scientific problem that your paper either solves or addresses. You should draw the reader in and make them want to read the rest of the paper.
%
%The next paragraphs in the introduction should cite previous research in this area. It should cite those who had the idea or ideas first, and should also cite those who have done the most recent and relevant work. You should then go on to explain why more work was necessary (your work, of course.)
% 
%What else belongs in the introductory section (s) of your paper? 
%
%1.    A statement of the goal of the paper: why the study was undertaken, or why the paper was written. Do not repeat the abstract. 
%
%2.   Sufficient background information to allow the reader to understand the context and significance of the question you are trying to address. 
%
%3. Proper acknowledgement of the previous work on which you are building.
%Sufficient references such that a reader could, by going to the library, achieve
%a sophisticated understanding of the context and significance of the question.
%
%4.    The introduction should be focused on the thesis question(s).  All cited work should be directly relevant to the goals of the thesis.  This is not a place to summarize everything you have ever read on a subject.
%
%5.    Explain the scope of your work, what will and will not be included. 
%
%6.    A verbal `road map' or verbal `table of contents' guiding the reader to what lies ahead. 
%
%7.    Is it obvious where introductory material (`old stuff') ends and your contribution (`new stuff') begins? 
%
%Remember that this is not a review paper. We are looking for original work and interpretation/analysis by you. Break up the introduction section into logical segments by using subheads. 

%%% Local Variables:
%%% mode: latex
%%% TeX-master: "dissertation"
%%% End:
