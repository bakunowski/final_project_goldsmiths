\chapter{Conclusion}
\label{chapterlabel6}

In this report I presented a system capable of near real-time
parameter prediction for a granular synthesizer based on sound input,
consisting of three main modules:

\begin{itemize}
\item Granular Synthesizer, implemented using ``JUCE''
\item MFCC extractor in ``Essentia''
\item Neural networks in ``Keras''
\end{itemize}

These come together to create a standalone program, with focus on user
experience, and ease of use. As well as the overall correctness of
predicted sounds.

The overall performance of the program could be drastically improved,
by further enhancing the capabilities of each of the above mentioned
modules. Yet, I have managed to create a program helpful in creating
sounds, that has the potential to challange the popular use of
presets, and achieve predictions that differ from simply randomizing
values of the synthesizer, meeting all three of my project's
objectives, as determined by both quantative and qualitative analysis.

The predictions obtained from a microphone input, even in noisy
environments, were rated well by testers, and achieved a fairly good
score on the similiarity scale. Additionally, questionnaire
respondents deemed the software intuitive and easy to use. The
qualitative feedback received from the respondents suggests clear way
of improving the software, yet suggests that there is real interest in
using such tool in a studio environment.

A standalone synthesizer, capable of near real-time predictions, that
integrates machine learning solutions is sufficiently different from
alternatives, despite using realtively naive methods, and I submit
this as a justification for the work's value in relation to state of
the art literature.

%%% Local Variables:
%%% mode: latex
%%% TeX-master: "dissertation"
%%% End:
