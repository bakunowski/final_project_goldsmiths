\chapter{Results and Analysis}
\label{chapterlabel4}
\iffalse
    The results are actual statements of observations, including statistics, tables and graphs.
    Indicate information on range of variation.
    Mention negative results as well as positive. Do not interpret results - save that for the discussion. 
    Lay out the case as for a jury. Present sufficient details so that others can draw their own inferences and construct their own explanations. 
    Use S.I. units (m, s, kg, W, etc.) throughout the thesis. 
    Break up your results into logical segments by using subheadings
    Key results should be stated in clear sentences at the beginning of paragraphs.  It is far better to say "X had significant positive relationship with Y (linear regression p<0.01, r^2=0.79)" then to start with a less informative like "There is a significant relationship between X and Y".  Describe the nature of the findings; do not just tell the reader whether or not they are significant. 

Note: Results vs. Discussion Sections
Quarantine your observations from your interpretations. The writer must make it crystal clear to the reader which statements are observation and which are interpretation. In most circumstances, this is best accomplished by physically separating statements about new observations from statements about the meaning or significance of those observations. Alternatively, this goal can be accomplished by careful use of phrases such as "I infer ..." vast bodies of geological literature became obsolete with the advent of plate tectonics; the papers that survived are those in which observations were presented in stand-alone fashion, unmuddied by whatever ideas the author might have had about the processes that caused the observed phenomena.
 
How do you do this? 

    Physical separation into different sections or paragraphs.
    Don't overlay interpretation on top of data in figures. 
    Careful use of phrases such as "We infer that ".
    Don't worry if "results" seem short.

Why? 

    Easier for your reader to absorb, frequent shifts of mental mode not required. 
    Ensures that your work will endure in spite of shifting paradigms.
\fi